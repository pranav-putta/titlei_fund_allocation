\begin{abstract}
    The decennial census is the primary source of statistical information about the US population.
    In 2020, the US Census Bureau applied disclosure avoidance methodologies to provide privacy guarantees
    over the data products. In this work, we study how the application of differential privacy affects
    fairness and bias in federal fund allocation programs. This paper provides a deeper understanding
    of the fairness issues arising from differentially private data, what factors produce fairness
    problems, and we propose techniques to mitigate biases and bound unfairness. We offers formal
    definitions of fairness and bounded fairness, analyze the structure of decision-making problems,
    and provide guidelines to alleviate negative fairness effects.
\end{abstract}
