\section*{Introduction}

Agencies, such as the U.S.~Census Bureau, release data sets and
statistics about groups of individuals that are then used as inputs to
a number critical decision processes. For example, the census data is
used to decide whether a jurisdiction must provide language assistance
during elections, Title I fund allocation in education \cite{pujol:20}
and to establish national level COVID-19 vaccination distribution plans
for states and jurisdictions \cite{covid}.
The resulting decisions can have significant societal, economic, and medical
consequences for participating individuals.

In many cases, the released data
contain sensitive information and their privacy is strictly regulated.
For example, in the U.S., the census data is regulated under Title 13
\cite{title13}, which requires that no individual be identified from
any data release by the Census Bureau. In Europe, data release are
regulated according to the \emph{General Data Protection Regulation}
\cite{GDPR}, which addresses the control and transfer of personal data.

Statistical agencies thus release \emph{privacy-preserving} data and
statistics that conform to privacy and confidentiality requirements.
In the U.S., a small number of decisions, such as congressional
apportionment, are taken using unprotected true values, but the vast
majority of decisions rely on privacy-preserving data. Of particular
interest are resource allocation decisions relying on the U.S.~Census
Bureau data, since the bureau will release several privacy-preserving
data products using the framework of \emph{Differential Privacy} \cite{abowd2018us}
for their 2020 release.

In particular, in 2020 the Census Bureau implemented a new privacy preserving framework to release privately hierarchical
statistical data, the Top Down algorithm. The algorithms works by firstly splitting the given privacy budget $\epsilon$ to six
hierarchical levels (nation, state, county, tract, block, group—block). Then in the second step, a post-processing step
is applied to make sure the noisy counts are consistent, e.g., the counts should be non-negative.
However, \cite{pujol:20} empirically showed that differential privacy may have a disparate impact on several resource
allocation problems. The noise introduced by the privacy mechanism may result in decisions that impact various groups differently.
Unfortunately, the paper did not provide a deep understanding why this behavior happens and any mitigation to resolve the issue.
This paper builds on these observations and provides a step towards a deeper understanding of the fairness issues arising when
differentially private data is used as input to several resource allocation problems.
	{\em One of its main results
is to prove that several allotment problems and decision rules with significant societal impact (e.g., the allocation of
educational funds, the decision to provide minority language assistance on election ballots, or the distribution of
COVID-19 vaccines) exhibit inherent unfairness when applied to a differentially private release of the census
data.} To counteract this negative results, the paper examines the conditions under which decision making is fair when
using differential privacy, and techniques to bound unfairness. The paper also provides a number of mitigation
approaches to alleviate biases introduced by differential privacy on such decision making problems. More specifically,
we make the following contributions:

\begin{enumerate}[leftmargin=*,labelsep=2pt,itemsep=0pt,parsep=2pt,topsep=2pt]

	\item We formally defines notions of fairness and bounded fairness for decision making
	subject to privacy requirements.

	\item We examine the roots of the induced unfairness by analyzing the structure
	of the decision making problems.

	\item We propose several guidelines to mitigate the negative
	fairness effects of the decision problems studied.
\end{enumerate}

To the best of the authors' knowledge, this is the first study that
attempt at characterizing the relation between differential privacy
and fairness in decision problems. All proofs are reported in the
appendix. [Is this still true?]
