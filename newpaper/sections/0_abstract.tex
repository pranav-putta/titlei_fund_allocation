\begin{abstract}
	The decennial census is a primary source of data for the US government to make critical decisions. For example,
	132 programs used Census Bureau data to distribute more than \$675 billion in funds during fiscal year
	2015. In order to ensure the published census data does not reveal individual information, the Census Bureau adopts
	the differential privacy (DP) technique. In particular, in 2020, the Census Bureau implemented the Top Down
	algorithm which release statistical data from top (nation) to bottom hierarchical level (block). However, it was
	recently observed that the DP outcomes can introduce biased outcomes, especially for minority groups. In this paper,
	we analyze the reasons for these disproportionate impacts and proposes guidelines to mitigate these effects. We
	focus on two aspects that can produce the fair outcomes: (1) shape of allocation function and (2) impact of
	post-processing steps.
\end{abstract}