\significancestatement{This research addresses a critical gap in understanding the
relationship between privacy-preserving data release methods,
	particularly differential privacy, and the fairness of consequen-
	tial societal decisions derived from such data. By investigating
	the impacts of differential privacy on resource allocation, this
	study unveils the inherent potential for bias against various
	groups, which could detrimentally affect areas such as educa-
	tional funding, electoral assistance, and public health initiatives.
	In response to these findings, our work proposes guidelines
	to bound unfairness and foster equity, providing a roadmap for
	future applications of differential privacy in data-driven decision
	making. The insights gleaned from this study have implications
	not only for the application of privacy-preserving techniques,
	but also for broader conversations around data privacy, fairness,
	and social equity}
